% !TEX TS-program = pdflatex
% !TEX encoding = UTF-8 Unicode

% This is a simple template for a LaTeX document using the "article" class.
% See "book", "report", "letter" for other types of document.

\documentclass[11pt]{article} % use larger type; default would be 10pt

\usepackage[utf8]{inputenc} % set input encoding (not needed with XeLaTeX)

%%% Examples of Article customizations
% These packages are optional, depending whether you want the features they provide.
% See the LaTeX Companion or other references for full information.

%%% PAGE DIMENSIONS
\usepackage{geometry} % to change the page dimensions
\geometry{a4paper} % or letterpaper (US) or a5paper or....
% \geometry{margin=2in} % for example, change the margins to 2 inches all round
% \geometry{landscape} % set up the page for landscape
%   read geometry.pdf for detailed page layout information

\usepackage{graphicx} % support the \includegraphics command and options

% \usepackage[parfill]{parskip} % Activate to begin paragraphs with an empty line rather than an indent

%%% PACKAGES
\usepackage{booktabs} % for much better looking tables
\usepackage{array} % for better arrays (eg matrices) in maths
\usepackage{paralist} % very flexible & customisable lists (eg. enumerate/itemize, etc.)
\usepackage{verbatim} % adds environment for commenting out blocks of text & for better verbatim
\usepackage{subfig} % make it possible to include more than one captioned figure/table in a single float
% These packages are all incorporated in the memoir class to one degree or another...

%%% HEADERS & FOOTERS
\usepackage{fancyhdr} % This should be set AFTER setting up the page geometry
\pagestyle{fancy} % options: empty , plain , fancy
\renewcommand{\headrulewidth}{0pt} % customise the layout...
\lhead{}\chead{}\rhead{}
\lfoot{}\cfoot{\thepage}\rfoot{}
\usepackage{listings}


%%% SECTION TITLE APPEARANCE
\usepackage{sectsty}
\allsectionsfont{\sffamily\mdseries\upshape} % (See the fntguide.pdf for font help)
% (This matches ConTeXt defaults)

%%% ToC (table of contents) APPEARANCE
\usepackage[nottoc,notlof,notlot]{tocbibind} % Put the bibliography in the ToC
\usepackage[titles,subfigure]{tocloft} % Alter the style of the Table of Contents
\renewcommand{\cftsecfont}{\rmfamily\mdseries\upshape}
\renewcommand{\cftsecpagefont}{\rmfamily\mdseries\upshape} % No bold!

%%% END Article customizations

%%% The "real" document content comes below...

\title{CS 770: Assignment 5}
\author{Ronghao Yang\\ID: 20511820}
%\date{} % Activate to display a given date or no date (if empty),
         % otherwise the current date is printed 

\begin{document}
\maketitle

\section{Exercise 1}
\begin{lstlisting}[language=Octave]
function [C,A] = myDFT(f,X)
% discrete Fourier transform
% Output: C,A
% C contains the DFT coefficients
% A contains the DFT approxmiation
% Input: f,X
% f contains the function values
% X contains the X values which are to be approximated
    
    n = length(f);
    % n is the number of points
    
    C=zeros(1,n);
    % initialize the coefficients to 0
    
    i = sqrt(-1);
    % initialize i
    
    for k = 0:n-1
        for j = 0:n-1
            C(k+1) = C(k+1)+(1./n)*f(j+1)*exp(-2*pi*k*(j./n)*i);
        end
    end
    % Calculating the DFT coefficients
    
    N = length(X);
    A = zeros(1,N);
    for j = 0:N-1
        for k = 0:n-1
            A(j+1) = A(j+1) + C(k+1)*exp(2*pi*(k)*i*X(j+1));            
        end
    end
    % Calculating the DFT approximations
end
\end{lstlisting}
\section{Exercise 2}
\subsection{Exercise 2a}
$f(x)=x$
\subsection{Exercise 2b}
$f(x)=exp(cos(2\pi x))$
\subsection{Exercise 2c}
$f(x)=((x-0.5)/0.5)^{2}$
\subsection{Exercise 2d}
$f(x)=((x-0.5)/0.5)^{m}$

\section{Exercise 3}
For Fourier series,\\\\
\centerline{$a_{k}$ = $2\int_{0}^{1}f(x)cos(2\pi k x)dx$, $b_{k}$ = $2\int_{0}^{1}f(x)sin(2\pi k x)dx$}
\subsection{Exercise 3a}
For $f(x)=(cos(8\pi x))^{4}$, \\\\
\centerline{$a_{k}$ = $2\int_{0}^{1}(cos(8\pi x))^{4}cos(2\pi k x)dx$, $b_{k}$ = $2\int_{0}^{1}(cos(8\pi x))^{4}sin(2\pi k x)dx$}\\\\
\centerline{Since $cos(2x) = 2cos(x)^{2}-1$}\\\\
\centerline{Then $cos(8\pi x)^{4}$ = $\frac{1}{4}(\frac{cos(32\pi x)}{2}+2cos(16\pi x)+1)$}\\\\
By the orthogonality property,\\\\
\centerline{$a_{8} =\frac{1}{4}$, $a_{16} = 1$, All $b_{k}$ is 0}
\subsection{Exercise 3b}
For $f(x)=x$,\\\\
%\centerline{$C_{k}$ = $\int_{0}^{1}xe^{-2\pi i k x}$}\\\\
%\centerline{Let $u = x, \frac{dv}{dx} = e^{-2\pi i k x}$}\\\\
%\centerline{Then $du = 1, v = \frac{1}{-2\pi i k}e^{-2\pi i k x}$}\\\\
%\centerline{Then $\int xe^{-2\pi i k x}$  = $\frac{x}{-2\pi i k}e^{-2\pi i k x}-\int \frac{1}{-2\pi i k}e^{-2\pi i k x}$}\\\\
%\centerline{$\int xe^{-2\pi i k x}$ = $\frac{x}{-2\pi i k}e^{-2\pi i k x}-\frac{1}{-4\pi^{2}k^2}e^{-2\pi i k x}$}\\\\
%\centerline{Since $C_{k}$ = $\int_{0}^{1}xe^{-2\pi i k x}$}\\\\
%\centerline{Then $C_{k}$ = $(\frac{1}{-2\pi i k}e^{-2\pi i k}-\frac{1}{-4\pi^{2}k^2}e^{-2\pi i k}) - (0 - \frac{1}{-4\pi^{2}k^2})$}\\\\
%\centerline{$C_{k}$ = $\frac{•}{-4\pi^{2}k^2}$}\\\\
\centerline{$a_{k}$ = $\int_{0}^{1}xcos(2\pi k x)dx$, $b_{k}$ = $\int_{0}^{1}xsin(2\pi k x)dx$}\\\\

\section{Exercise 4}
When the function is periodic on $[a,b]$ instead of being periodic on the interval of $[0,1]$\\\\
\centerline{Let $I = b-a$, then $C_{k} = \sum_{0}^{n-1}f(x_{n})e^{-2\pi i \frac{k}{I}x_{n}}$}\\\\
When approximating the original function, we have \\\\
\centerline{$f(x)$ = $\sum_{0}^{k-1}C_{k}e^{2\pi i \frac{k}{I}x}$}\\\\
For example,\\\\
\centerline{Let $f(x) = x-5$ on interval $[5,10]$}\\\\

\section{Exercise 5}
\section{Exercise 6}
\subsection{8.1}
The phone number is 
\subsection{8.2}



\end{document}
