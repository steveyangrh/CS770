% !TEX TS-program = pdflatex
% !TEX encoding = UTF-8 Unicode

% This is a simple template for a LaTeX document using the "article" class.
% See "book", "report", "letter" for other types of document.

\documentclass[11pt]{article} % use larger type; default would be 10pt

\usepackage[utf8]{inputenc} % set input encoding (not needed with XeLaTeX)

%%% Examples of Article customizations
% These packages are optional, depending whether you want the features they provide.
% See the LaTeX Companion or other references for full information.

%%% PAGE DIMENSIONS
\usepackage{geometry} % to change the page dimensions
\geometry{a4paper} % or letterpaper (US) or a5paper or....
% \geometry{margin=2in} % for example, change the margins to 2 inches all round
% \geometry{landscape} % set up the page for landscape
%   read geometry.pdf for detailed page layout information

\usepackage{graphicx} % support the \includegraphics command and options

% \usepackage[parfill]{parskip} % Activate to begin paragraphs with an empty line rather than an indent

%%% PACKAGES
\usepackage{booktabs} % for much better looking tables
\usepackage{array} % for better arrays (eg matrices) in maths
\usepackage{paralist} % very flexible & customisable lists (eg. enumerate/itemize, etc.)
\usepackage{verbatim} % adds environment for commenting out blocks of text & for better verbatim
\usepackage{subfig} % make it possible to include more than one captioned figure/table in a single float
% These packages are all incorporated in the memoir class to one degree or another...

%%% HEADERS & FOOTERS
\usepackage{fancyhdr} % This should be set AFTER setting up the page geometry
\pagestyle{fancy} % options: empty , plain , fancy
\renewcommand{\headrulewidth}{0pt} % customise the layout...
\lhead{}\chead{}\rhead{}
\lfoot{}\cfoot{\thepage}\rfoot{}

%%% SECTION TITLE APPEARANCE
\usepackage{sectsty}
\allsectionsfont{\sffamily\mdseries\upshape} % (See the fntguide.pdf for font help)
% (This matches ConTeXt defaults)
\usepackage{amsmath}
\usepackage{systeme}
\usepackage{listings}

%%% ToC (table of contents) APPEARANCE
\usepackage[nottoc,notlof,notlot]{tocbibind} % Put the bibliography in the ToC
\usepackage[titles,subfigure]{tocloft} % Alter the style of the Table of Contents
\renewcommand{\cftsecfont}{\rmfamily\mdseries\upshape}
\renewcommand{\cftsecpagefont}{\rmfamily\mdseries\upshape} % No bold!

%%% END Article customizations

%%% The "real" document content comes below...

\title{CS770: Assignment 4}
\author{Ronghao Yang\\ID: 20511820}
%\date{} % Activate to display a given date or no date (if empty),
         % otherwise the current date is printed 

\begin{document}
\maketitle

\section{Question 1}
\subsection{Question 1a}
For cubic spline, let the function $S(x)$ define the spline function\\\medskip
\centerline{where $a = x_{0} < x_{1} < x_{2} < x_{3} < ......< x_{n} = b$}
 
\begin{equation}
  S(X)=\left\{
  \begin{array}{@{}ll@{}}
    S_{0}(x), & \text{}\ x_{0}<x<x_{1} \\
    S_{1}(x), & \text{}\ x_{1}<x<x_{2}\\
    ......\\
    S_{i}(x), & \text{} \ x_{i}<x<x_{i+1}\\
    ......\\
    S_{n-1}(x), & \text{}\ x_{n-1}<x<x_{n}
  \end{array}\right.
\end{equation} 
\centerline{Where each $S_{i}(x)$ has degree 3 in this case.}\\
In cubic spline, $S(x)$ satisfies
\[
\systeme*{
S_{i}(x_{i})=S_{i+1}(x_{i}),
S_{i}'(x_{i})=S_{i+1}'(x_{i}), 
S_{i}''(x_{i})=S_{i+1}''(x_{i}),
S_{i}(x_{i}) = y_{i}
}
\]
\centerline{Where each $i = 0, 1, 2, ......, n-2$}\\
By the definition of natural cubic spline, we have two additional constraints,
\[
\systeme*{
S_{0}''(x_{0})=0,
S_{n-1}''(x_{n-1})=0
}
\]
\subsection{Question 1b}
\begin{lstlisting}[language=Octave]
function [coeffs] = nSpline(X,y)
%This function returns the coefficients of the natural cubic spline
%X and y are the input points where f(X(i)) = y(i)
%Each spline function on each interval has degree 3

%Si = a+bx+cx^2+dx^3
%We have n such Si's, where is n = length(X)-1
%coeffs should be [a1;b1;c1;d1;a2;b2;.....;an;bn;cn;dn]
%coeffs is a 4n by 1 vector
    
    numP = length(X);
    %numP is the number of points
    n =numP - 1;
    %n is the number of spline functions
    
    X = [];
    %initialize the matrix to be empty
    A = [];
    %A contains all the known values
    %X*coeffs = A
    
    %we construct the matrix using for loop
    for i = 1:numP
        a = (i-1)*4+1;
        b = a+1;
        c = b+1;
        d = c+1;
        %a,b,c,d are indices for the convinience of calculation
        
        if i==1
            tempX = zeros(1,4*n);
            tempX(1,c) = 2;
            tempX(1,d) = 6*X(i);
            X = [X;tempX];
            A = [A;0];
            tempX = zeros(1,4*n);
            tempX(1,a) = 1;
            tempX(1,b) = X(i);
            tempX(1,c) = X(i)^2;
            tempX(1,d) = X(i)^3;
            X = [X;tempX];
            A = [A;y(i)];
        end
        
        
        if i==numP
            tempX = zeros(1,4*n);
            tempX(1,c-4) = 2;
            tempX(1,d-4) = 6*X(i);
            X = [X;tempX];
            A = [A;0];
            tempX = zeros(1,4*n);
            tempX(1,a-4) = 1;
            tempX(1,b-4) = X(i);
            tempX(1,c-4) = X(i)^2;
            tempX(1,d-4) = X(i)^3;
            X = [X;tempX];
            A = [A;y(i)];
        end
        %these the special end points contraints for natural cubic constraint
        
        if i>1 && i<numP
            tempX = zeros(1,4*n);
            tempX(1,c-4) = 2;
            tempX(1,d-4) = 6*X(i);
            X = [X;tempX];
            A = [A;0];
            tempX = zeros(1,4*n);
            tempX(1,a-4) = 1;
            tempX(1,b-4) = X(i);
            tempX(1,c-4) = X(i)^2;
            tempX(1,d-4) = X(i)^3;
            X = [X;tempX];
            A = [A;y(i)];
        end
        %this is the constraint for S(xi) = yi
        
        if i>1 && i<numP
            tempX = zeros(1,4*n);
            tempX(1,b-4) = 1;
            tempX(1,c-4) = 2*X(i);
            tempX(1,d-4) = 3*X(i)^2;
            tempX(1,b) = -1;
            tempX(1,c) = -2*X(i);
            tempX(1,d) = -3*X(i)^2;
            X = [X;tempX];
            A = [A;0];
            %this is the constraint for Si'(xi) = Si+1'(xi)
            
            tempX = zeros(1,4*n);
            tempX(1,c-4) = 2;
            tempX(1,d-4) = 6*X(i);
            tempX(1,c) = -2;
            tempX(1,d) = -6*X(i);
            X = [X;tempX];
            A = [A;0];
            %this is the constraint for Si''(xi) = Si+1''(xi)
        end
    end
    
    coeffs = X\A;
    
end
\end{lstlisting}

\section{Question 2}
\section{Question 3}
\subsection{Question 3a}
\subsection{Question 3b}

\section{Question 4}
\section{Question 5}
\subsection{Question 5a}
\subsection{Question 5b}

\section{Question 6}
\section{Question 7}

\end{document}
